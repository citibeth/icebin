\section{Four Contracts}

We think of the coupling as going direclty between GCM and Ice Model.
However, this is not the case: not only do the GCM and Ice Model run
on different grids, they might also produce/accept different
variables, or variables in different units.  For this reason, we can
think of the process of transferring data from GCM to Ice Model (or
vice versa) as a multi-step process.

Suppose that the GCM produces $n$ outputs on the elevation grid $E$,
$\bm{D}^E_1 \ldots \bm{D}^E_n$.  We can think of this as an
$n$-dimensional vector $\bm{\mathcal{D}}^E$.  The following two steps
are involved in transferring to the ice model.  They may be carried
out in either order, since they are both linear:

\begin{enumerate}

\item Regrid $\bm{\mathcal{D}}^E$ to $\bm{mathcal{D}}^I$.

\item Take a linear combination of the $\bm{D}^I$ variables to produce $\bm{E}_I$ variables, which will be the ice model input.  The linear combination is needed for the following unit conversion scenarios:
 \begin{enumerate}
 \item Simple unit conversion (eg: \si{\kilo\gram\per\square\meter} = $\rho_w$ \si{\milli\meter} water equivalent)
 \item Offset unit conversion (eg: \si{K} = \si{\degC} + 273.15)
 \item Linear combination required for enthalpy conversions.
 \end{enumerate}

For this reason, coupling involves four contracts: GCM-out, Ice-in,
Ice-out, GCM-in, and two linear transformations converting GCM-out
$\rightarrow$ Ice-in, and Ice-out $\rightarrow$ GCM-in.

The contracts are set in the following places:

\begin{enumerate}
 \item[GCM-out:] {\tt GCMCoupler_ModelE.cpp}
 \item[Ice-in:] {\tt contracts/modele_pism.cpp}
 \item[Ice-out:] {\tt contracts/modele_pism.cpp}
 \item[GCM-in:] {\tt LANDICE_COM.f}
\end{}
